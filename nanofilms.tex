\documentclass[14pt]{extarticle}


\usepackage[english, russian]{babel}
\usepackage{fontspec}
        \setmainfont{Times New Roman}

\renewcommand{\bf}{\textbf}

\usepackage{xcolor}
\usepackage{hyperref}
    \hypersetup{linkcolor=black, urlcolor=black, colorlinks=true, citecolor=black}

\usepackage{graphicx}

\usepackage{tikz}

\usepackage{amsmath, amssymb}


\usepackage[left=3cm, right=1.5cm, top=2cm, bottom=2cm]{geometry}

\usepackage{import}
\usepackage{xifthen}
\usepackage{pdfpages}
\usepackage{transparent}

% \newcommand{\incfig}[1]{%
%     \def\svgwidth{\columnwidth}
%     \import{./картинки/}{#1.pdf_tex}
% }

% \graphicspath{{картинки/}}

\renewcommand{\contentsname}{\normalfont \normalsize Содержание}

% Ставим 1.5 интервал согласно требованиям
\linespread{1.3}

\usepackage{titlesec}

\titleformat{\section}{\normalfont\normalsize \large\bfseries}{\normalsize\thesection}{14pt}{\normalfont\normalsize \bfseries}

\titleformat{\subsection}{\normalfont\normalsize \large\bfseries}{\normalsize\thesubsection}{14pt}{\normalfont\normalsize \bfseries}












\begin{document}


\tableofcontents





\section{Введение}

Что такое углеродные наноплёнки?
Углеродные нанопленки представляют собой тонкие пленки материалов на основе углерода, которые наносятся на подложку с использованием различных методов, таких как химическое осаждение из паровой фазы (CVD) или физическое осаждение из паровой фазы (PVD). Эти пленки обычно имеют толщину от нескольких нанометров до нескольких микрометров и обладают уникальными свойствами, такими как высокая механическая прочность, отличная электропроводность и хорошая термическая стабильность. Углеродные нанопленки находят применение в самых разных областях, включая электронику, хранение энергии, биомедицинскую технику и покрытия.

% https://i.imgur.com/7OPAOzW.png


% ![](https://i.imgur.com/kGdqi8w.png)


























\end{document}