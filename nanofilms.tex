\documentclass[14pt]{extarticle}


\usepackage[english, russian]{babel}
\usepackage{fontspec}
        \setmainfont{Times New Roman}

\renewcommand{\bf}{\textbf}

\usepackage{xcolor}
\usepackage{hyperref}
    \hypersetup{linkcolor=black, urlcolor=black, colorlinks=true, citecolor=black}

\usepackage{graphicx}

\usepackage{tikz}

\usepackage{amsmath, amssymb}


\usepackage[left=3cm, right=1.5cm, top=2cm, bottom=2cm]{geometry}

\usepackage{import}
\usepackage{xifthen}
\usepackage{pdfpages}
\usepackage{transparent}

% \newcommand{\incfig}[1]{%
%     \def\svgwidth{\columnwidth}
%     \import{./картинки/}{#1.pdf_tex}
% }

% \graphicspath{{картинки/}}

\renewcommand{\contentsname}{\normalfont \normalsize Содержание}

% Ставим 1.5 интервал согласно требованиям
\linespread{1.3}

\usepackage{titlesec}

\titleformat{\section}{\normalfont\normalsize \large\bfseries}{\normalsize\thesection}{14pt}{\normalfont\normalsize \bfseries}

\titleformat{\subsection}{\normalfont\normalsize \large\bfseries}{\normalsize\thesubsection}{14pt}{\normalfont\normalsize \bfseries}












\begin{document}


\tableofcontents


% "Исследование газочувствительных свойств наноуглеродных пленок"


\section{Введение}

\subsection{Что такое наноплёнки?}

Под наноструктурными (нанокристаллическими, нанокомпозитными, нанофазными, нановолокнистыми и т.д.) материалами понимают материалы, в которых размеры основных структурных элементов (кристаллитов, волокон, слоѐв, пор) не превышают 100 нм, по крайней мере, в одном направлении. Объекты, размер которых составляет 0,1нм (порядок размеров отдельных атомов) ÷ 100 нм (порядок размеров крупных молекул), являются предметом изучения для нанотехнологии, бурно развивающейся в последние несколько десятков лет.

Научные исследования нанообъектов были начаты ещѐ в 1856 – 1857 гг. М. Фарадеем – при исследовании свойств коллоидных растворов высокодисперсного золота и тонких плѐнок он заметил, что их цвет изменяется при изменении размеров его частиц.










\subsection{Особенности наноплёнок, размерные эффекты}

Все наноразмерные материалы, в том числе и плёнки, практически бездефектны, и поэтому сильно отличаются по свойствам от соответствующих макроматериалов. Так, в книге Дж. Гордона «Почему мы не проваливаемся сквозь пол» (М.: Мир, 1971. – 272 с.) отмечается практически близкая к теоретической механическая прочность на растяжение нитевидных кристаллов – усов, вне зависимости от химической природы кристалла и метода его выращивания.

Физические свойства низкоразмерных структур сильно отличаются от свойств этих же, но макроскопических систем. В настоящее время низкоразмерные структуры, к которым относятся тонкопленочные структуры, являющиеся двухмерными объектами, активно исследуются как на фундаментальном уровне, так и с прикладными целями. Важнейшим фактором, определяющим свойства тонкопленочных структур, выступают релаксационные процессы, протекающие в самих пленках и на границах раздела. В самих пленках, в первую очередь, эволюция структуры связана с процессами релаксации свободного объема. В слоистых системах релаксация свободного объема, кроме того, сопровождается процессами на границе раздела.










\subsection{Наноплёнки: современные исследования и достижения}



\subsection{Углеродные наноплёнки и их свойства}


Что такое углеродные наноплёнки?
Углеродные нанопленки представляют собой тонкие пленки материалов на основе углерода, которые наносятся на подложку с использованием различных методов, таких как химическое осаждение из паровой фазы (CVD) или физическое осаждение из паровой фазы (PVD). Эти пленки обычно имеют толщину от нескольких нанометров до нескольких микрометров и обладают уникальными свойствами, такими как высокая механическая прочность, отличная электропроводность и хорошая термическая стабильность. Углеродные нанопленки находят применение в самых разных областях, включая электронику, хранение энергии, биомедицинскую технику и покрытия.

Нанопленки углерода (carbon nanofilms) представляют собой тонкие пленки углерода, толщина которых составляет от нескольких до нескольких десятков нанометров. Они могут быть получены различными способами, включая химическое осаждение, физический осадок и пиролиз.
Нанопленки углерода обладают рядом уникальных свойств, таких как высокая электропроводность, прочность, жесткость и устойчивость к коррозии. Они также имеют большую поверхностную площадь, что делает их полезными для различных приложений в области электроники, катализа, сенсорики и биомедицины.

Особый интерес вызывают углеродные пленки в связи с тем, что атомы углерода могут образовать несколько кристаллических модификаций, среди которых есть хорошо известные структуры графита и алмаза. Есть и менее известные структуры в виде замкнутых полых структур, получивших название фуллерена С60, содержащего 60 атомов углерода, и модификации С60 с другим содержанием углерода и моноатомных плоскостей.







\subsection{Исследования газочувствительности углеродных наноплёнок}

Ещё в 2007 году группа Новоселова [22] сообщила о первом датчике газа на основе графена, который продемонстрировал, что датчики микрометрового размера, сделанные из графена, способны обнаруживать отдельные молекулы газа, которые прикрепляются к поверхности графена или отделяются от нее. Они показали, что адсорбированные молекулы изменяют локальную концентрацию носителей в графене один электрон за одним, что приводит к ступенчатым изменениям сопротивления. Вызванные газом изменения удельного сопротивления имели разную величину для разных газов, и знак изменения указывал на то, был ли газ акцептором электронов (например, NO 2 , H 2 O, йод) или донором электронов (например, NH 3, СО, этанол). Это исследование открыло перед исследователями новые возможности для разработки газовых сенсоров на основе графена [26]. Взаимодействие между листами графена и адсорбатами может варьироваться от слабой ван-дер-ваальсовой до сильной ковалентной связи. Все эти взаимодействия изменяют электронную структуру графена, которую легко контролировать удобными электронными методами. Хилл и др. [27] предположили, что уровень взаимодействия между целевыми молекулами газа/пара может достигать нижнего предела даже одной молекулы, т.е. высокой чувствительности даже при низких концентрациях газа [28] , [29] .

Одно из исследований на тему газочувствительности нанопленок углерода было опубликовано в журнале Carbon в 2018 году. В этом исследовании авторы изучали эффект изменения температуры на газочувствительность нанопленок углерода к аммиаку.
Другое исследование, опубликованное в журнале Sensors and Actuators B: Chemical в 2019 году, рассматривало газочувствительность нанопленок углерода к оксиду углерода. В этом исследовании авторы использовали различные методы обработки нанопленок углерода, чтобы определить оптимальные условия для достижения максимальной газочувствительности.
Третье исследование, опубликованное в журнале ACS Applied Materials \& Interfaces в 2020 году, исследовало газочувствительность нанопленок углерода к диоксиду серы. В этом исследовании авторы использовали нанопленки углерода, полученные методом химического осаждения, и изучили их свойства при различных температурах и концентрациях газа.
В целом, исследования газочувствительности нанопленок углерода показывают потенциал этого материала для создания высокочувствительных датчиков газовых смесей. Однако, как отмечается в этих исследованиях, дальнейшие исследования необходимы для определения оптимальных условий производства и использования нанопленок углерода в таких приложениях.



\subsection{Перспективы и возможности}

% https://i.imgur.com/7OPAOzW.png


% ![](https://i.imgur.com/kGdqi8w.png)


























\end{document}